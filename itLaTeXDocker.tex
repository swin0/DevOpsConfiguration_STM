\chapter{LaTeX Docker \\ 
\small{\textit{-- Tamara Gonzalez Ibarra, Michelle Elias Flores, Sydney Winstead}}}
\index{LaTeX Docker}
\index{Chapter!itLaTeXDocker}
\label{Chapter::itLaTeXDocker}

\section{Overview}

This chapter demonstrates how to create a minimal Docker container capable of compiling \LaTeX{} documents using \texttt{TeX Live}. This setup mimics how Overleaf performs compilations in isolated environments. By using Docker, we can ensure reproducibility, portability, and isolation — essential for consistent document builds across different systems.

\section{Requirements}

Before proceeding, we will ensure the Docker is installed and running on our system by running:

\begin{minted}[fontsize=\small, bgcolor=lightgray!10]{bash}
docker --version
\end{minted}

We will also need a simple \LaTeX{} document to compile, and a Dockerfile that defines the build environment.

\section{Minimal Project Structure}

We start with a simple directory structure:

\begin{minted}[fontsize=\small, bgcolor=lightgray!10]{text}
latex-docker/
├── Dockerfile
└── main.tex
\end{minted}

\section{The LaTeX Document}

The following is a minimal \texttt{main.tex} document that we will compile inside the container:

\begin{minted}[fontsize=\small, bgcolor=lightgray!10]{latex}

\begin{document}
\section*{Hello from Docker \& TeX Live!}
\lipsum[1]
\end{document}
\end{minted}

\section{Dockerfile Explanation}

The Dockerfile defines the environment needed to build and compile the document. We use a lightweight base image (Ubuntu) and install only the necessary TeX Live components.

\begin{minted}[fontsize=\small, bgcolor=lightgray!10]{docker}
# Use a minimal base image
FROM ubuntu:22.04

# Set a noninteractive mode for installation
ENV DEBIAN_FRONTEND=noninteractive

# Update packages and install TeX Live
RUN apt-get update && \
    apt-get install -y --no-install-recommends \
        texlive-latex-base \
        texlive-latex-recommended \
        texlive-latex-extra \
        texlive-fonts-recommended \
        latexmk && \
    apt-get clean && \
    rm -rf /var/lib/apt/lists/*

# Set the working directory
WORKDIR /data

# Command to compile LaTeX documents
# This runs latexmk to automatically detect and compile .tex files
ENTRYPOINT ["latexmk", "-pdf", "-interaction=nonstopmode"]
\end{minted}

\section{Building the Docker Image}

Build the image with the following command from the same directory as your Dockerfile:

\begin{minted}[fontsize=\small, bgcolor=lightgray!10]{bash}
docker build -t latex-docker .
\end{minted}

This command creates a Docker image named \texttt{latex-docker} that contains all required dependencies to compile \LaTeX{} documents.

\section{Compiling the Document}

Once built, you can compile your document by mounting your current directory and specifying your \texttt{.tex} file:

\begin{minted}[fontsize=\small, bgcolor=lightgray!10]{bash}
docker run --rm -v $(pwd):/data latex-docker main.tex
\end{minted}

After execution, you should find a compiled \texttt{main.pdf} in your directory.

\section{Conclusion}

Using Docker with TeX Live provides a reproducible and portable environment for \LaTeX{} document compilation. This approach mirrors Overleaf's backend concept — an isolated container environment that ensures consistent and dependency-free document builds across systems.




